\documentclass{article}

% paquete para imágenes y figuras del proyecto
\usepackage{graphicx}
% paquete para la bibliografía
\usepackage[backend=bibtex,sorting=none]{biblatex}
\bibliography{MNIG_to_FMMSP_Tatiana_Porras}

% metadata del proyecto
\title{MNIG to FMMSP Tatiana Porras}
\date{2020-03-13}
\author{Tatiana Porras}

% definición de variables
\def\direcM{LINDSAY ÁLVAREZ POMAR}
\def\direc{Lindsay Álvarez Pomar}
\def\algmod{MNIG\_to\_FMMSP} % algmod es por algoritmo modelo
\makeatletter
\let\dateVar\@date
\makeatother

% tesis:
\begin{document}

% página de título
\begin{titlepage}
    \begin{center}
    ALGORITMO VORAZ ITERATIVO CON MULTI-VECINDAD APLICADO AL PROBLEMA DE SECUENCIACIÓN DIFUSO MULTIPRODUCTO Y MULTIETAPAS
    \par \bigskip
    Anteproyecto de Grado
    \par \vspace{5cm}
    Tatiana Porras Cortes
    \par \medskip
    Correo \hspace{1cm} tatiporras96@gmail.com
    \par \bigskip
    LINDSAY ÁLVAREZ POMAR \par Director del trabajo de grado
    \par \vspace{1cm}
    \begin{figure}[h!]
        \begin{center}
        \includegraphics[width=6cm]{EscudoUD1.png}
        \end{center}
    \end{figure}
    \par UNIVERSIDAD DISTRITAL FRANCISCO JOSÉ DE CALDAS
    \par FACULTAD DE INGENIERÍA
    \par PROYECTO CURRICULAR DE INGENIERÍA INDUSTRIAL
    \par Bogotá D.C., Colombia. \dateVar
    \end{center}
\end{titlepage}

\renewcommand{\contentsname}{Tabla de Contenido}

\tableofcontents

\setcounter{section}{-1}

% secciones del proyecto
\section{TÍTULO}

"ALGORITMO VORAZ ITERATIVO CON MULTI-VECINDAD APLICADO AL PROBLEMA DE
SECUENCIACIÓN DIFUSO MULTIPRODUCTO Y \linebreak MULTIETAPAS"

\newpage

\section{RESUMEN}

El Algoritmo Voraz Iterativo con Multi-Vecindad aplicado al Problema de
Secuenciación Difuso Multiproducto y Multietapas (\algmod\ por sus siglas
en inglés) no ha sido tratado en la literatura científica internacional.
En este trabajo se explora esta combinación nueva. Para ello, se modificará
el algoritmo MNIG de modo que pueda ser utilizado para resolver el modelo FMMSP.

\vspace{\baselineskip}
Al final se reportarán los resultados del algoritmo aplicado a varias instancias 
del problema, y se compara dichos resultados con los de otros cuatro algoritmos 
que han sido aplicados al modelo FMMSP en la literatura.

\vspace{\baselineskip}
\textbf{Palabras Clave}: Algoritmo Voraz Iterativo, Multi-vecindad,
Difuso, Secuenciación, Multiproducto, Multietapas

\subsection{ABSTRACT}

The Multi-neighborhood Iterated Greedy algorithm applied to the Fuzzy 
Multiproduct Multistage Scheduling Problem (from now on \algmod)
has not been treated in the international scientific literature. This
work explores this new combination. For that, the MNIG algorithm will be
modified so that it can be used to solve the FMMSP model.

\vspace{\baselineskip}
At the end, the results of the algorithm will be reported, after applying it to 
several instances of the problem, and the results will be compared to those of 
another four algorithms that have been applied to the FMMSP model in the 
literature. 

\vspace{\baselineskip}
\par \textbf{Keywords}: Iterated Greedy, Multi-neighborhood, Scheduling, Fuzzy,
Multiproduct, Multistage

\section{INTRODUCCIÓN}

Mediante el presente trabajo de tesis de pregrado se pretende hacer un aporte
aunque pequeño al conocimiento. En la ingeniería industrial existen diversas
áreas en las que se puede hacer un aporte de este tipo. Existe el área de
investigación de operaciones, el área de gestión, mercadeo, higiene industrial,
seguridad y salud en el trabajo, ergonomía, etcétera. 

\vspace{\baselineskip}
Un aporte de nuevo conocimiento en pregrado es un aporte pequeño y muy
específico, detallado. Por ello se selecciona una de las áreas de conocimiento
de la ingeniería industrial y dentro de esa área se elige un tema en particular.
De ese tema elegido se trabaja un detalle que no haya sido estudiado con
anterioridad. 

\vspace{\baselineskip}
En el presente trabajo se eligió el tema de secuenciación, mejor conocido por
su nombre en inglés como "scheduling". Dentro de este tema se revisó el estado
del arte, y se encontró un modelo de scheduling que ha sido poco estudiado, y
por aparte se encontró un algoritmo que nunca ha sido aplicado al modelo, pero
que podría ser aplicado. El modelo encontrado \autocite{modFMMSP} es llamado 
FMMSP que significa Fuzzy Multiproduct Multistage Scheduling Problem, que en 
español se podría traducir como Problema de Secuenciación Difuso Multiproducto 
y Multietapas. Es un modelo interesante de ser estudiado pues incluye números 
difusos, múltiples productos, y múltiples etapas. El algoritmo encontrado 
\autocite{algMNIG} es llamado MNIG que significa Multi-Neighborhood Iterated
Greedy algorithm. En español se traduce como algoritmo Voraz Iterativo con 
Multi-Vecindad. Se ha encontrado que este algoritmo sirve para resolver 
problemas de scheduling \autocite{algMNIG}. 

\vspace{\baselineskip}
Lo novedoso resulta en que dicho algoritmo jamás ha sido aplicado al modelo, a 
tal punto que para realizar este trabajo se hará necesario modificar el 
algoritmo, pues el algoritmo original no se puede aplicar directamente al 
modelo. No existen pruebas de que este algoritmo sea apropiado para este modelo.
Con el presente trabajo se propone comprobar que tan bueno es el algoritmo MNIG
aplicado al modelo FMMSP, al compararlo con otros cuatro algoritmos que ya han
sido aplicados al modelo FMMSP \autocite{modFMMSP}.

\section{PLANTEAMIENTO DEL PROBLEMA}

\section{OBJETIVOS}

\section{JUSTIFICACIÓN}

\section{ALCANCES}

\section{MARCO DE REFERENCIA}

\section{HIPÓTESIS}

\section{DISEÑO METODOLÓGICO}

\section{CRONOGRAMA}

\section{PRODUCTOS DEL PROYECTO}

\section{BIBLIOGRAFÍA}

\printbibliography

\end{document}